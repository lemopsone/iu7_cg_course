\section{Конструкторская часть}

В данном разделе описаны используемые поля расстояний со знаком, алгоритм марширования лучей, модель освещения сцены, а также структура программы.

\subsection{Требования к программному обеспечению}

Программа должна предоставлять следующую функциональность:

\begin{itemize}
	\item добавление объекта из набора и настройка его параметров;
	\item изменение положения камеры в пространстве;
	\item изменение характеристик источника света.
\end{itemize}

\subsection{SDF-функции}

Поля расстояний со знаком (SDF) используются для нахождения кратчайшего расстояния
от точки в пространстве до поверхности тела. Знак результата определяет положение точки относительно данного тела:

\begin{itemize}
  \item вне объекта, если $SDF(p) > \varepsilon$;
  \item на границе объекта, если $|SDF(p)| \le \varepsilon$;
  \item внутри объекта, если $SDF(p) < -\varepsilon$;
\end{itemize}
где $\varepsilon \rightarrow 0$ --- допустимая погрешность при работе с числами с плавающей точкой.

SDF функции определяются для конкретных видов тел. Так, SDF для сферы имеет вид:
\begin{equation}
  SDF_{sphere}(x, y, z) = \sqrt{(x - x_0) ^ 2 + (y - y_0) ^ 2 + (z - z_0) ^ 2)} - r,
  \label{eq:sdf_sphere}
\end{equation}
где
\begin{itemize}
  \item $C(x_0, y_0, z_0)$ --- координаты центра сферы;
  \item $r$ --- радиус сферы.
\end{itemize}
Фрактальные структуры имеют итеративную природу, и поэтому SDF для них зачастую задаются в итеративном виде. Так, на рисунке~\ref{img:mandelbulb_sdf_png}
представлена схема поля расстояния со знаком для оболочки Мандельброта.

\png{190mm}{mandelbulb_sdf_png}{Схема алгоритма SDF для оболочки Мандельброта}
\clearpage

\subsection{Схема алгоритма марширования лучей}

Алгоритм марширования лучей отрисовывает объекты сцены при помощи полей расстояния со знаком (SDF).
Марширование лучей предполагает итеративное перемещение точки вдоль направления обзора (от камеры к объекту)
и проверку результата на основе полученного из SDF-функции значения.

Схема алгоритма марширования лучей представлена на рисунке~\ref{img:raymarching_png}.

\png{250mm}{raymarching_png}{Схема алгоритма марширования лучей}

\clearpage

\subsection{Модель освещения сцены}

Для освещения сцены используется модель Фонга. Согласно этой модели,
освещенность сцены является суммой трех компонент:

\begin{itemize}
  \item \textit{ambient} --- фоновое освещение;
  \item \textit{diffuse} --- диффузное освещение;
  \item \textit{specular} --- бликовое освещение.
\end{itemize}

Общая формула расчета освещенности в точке представлена в формуле~(\ref{eq:phong}).

\begin{equation}
  I = k_a i_a + k_d i_d (\vec{N} * \vec{L}) + k_s i_s (\vec{N} * \vec{R}) ^ \alpha,
  \label{eq:phong}
\end{equation}
где
\begin{itemize}
  \item $k_a, k_d, k_s$ --- коэффициенты фонового, диффузного и бликового освещения соответственно;
  \item $i_a, i_d, i_s$ --- интенсивности фонового, диффузного и бликового освещения соотвественно;
  \item $\vec{N}$ --- вектор нормали к поверхности в точке;
  \item $\vec{L}$ --- падающий луч (направление на источник света);
  \item $\vec{R}$ --- отраженный луч;
\end{itemize}

В стандартном случае, для вычисления вектора нормали использую нормированную взвешенную сумму векторов нормали граней, которым эта точка принадлежит:

\begin{equation}
  \vec{N} = \frac{a_1\vec{n_1}+\dots+a_k\vec{n_k}}{||a_1\vec{n_1}+\dots+a_k\vec{n_k}||}
  \label{eq:normal_std}
\end{equation}

Этот метод вычисления нормали работает только в тех случаях, когда информация об объектах хранится в виде полигональной сетки,
и есть возможность определить, к каким граням приналдлежит указанная точка. Так как в алгоритме марширования лучей
обхекты задаются в аналитическом виде, и получить информацию о вершинах и гранях объекта не представляется возможным.
Исходя из факта, что градиент функции $F(x, y, z)$ коллинеарен её нормали, мы можем с высокой точностью аппроксимировать нормаль:
\begin{equation}
  \vec{N} = grad SDF =
  \left(\begin{array}{c}
      SDF(x + \Delta, y, z) - SDF(x - \Delta, y, z)\\
      SDF(x, y + \Delta, z) - SDF(x, y - \Delta, z)\\
      SDF(x, y, z + \Delta) - SDF(x, y, z - \Delta)\\
    \end{array}\right) ^ T,
  \label{eq:normal_grad}
\end{equation}

Схема алгоритма закраски по Фонгу представлена на рисунке~\ref{img:phong_png}.

\png{150mm}{phong_png}{Схема алгоритма закраски по Фонгу}
\clearpage

\subsection{Выбор структур данных}

В работе будут использованы следующие типы и структуры данных:

\begin{itemize}
	\item точка --- вектор из трех вещественных чисел соответствующих координатам по каждой из осей трехмерной декартовой системы координат;
	\item цвет --- вектор из трех целых чисел, принимающих значение в диапазоне $[0,255]$, соответствует цветовой модели RGB;
	\item сфера --- объект сцены, содержит радиус, цвет объекта;
  \item фрактал --- объект сцены, содержит степень, используемую в формуле фрактала, количество итераций для построения, радиус отсечения, цвет объекта;
	\item источник света --- содержит интенсивность и точку своего положения в пространстве;
	\item камера --- содержит точку своего положения в пространстве, угол обзора и тройку векторов, задающую направление камеры;
	\item сцена --- хранит в себе набор объектов, камеру, источник света.
\end{itemize}

\subsection{Структура программы}

Программа состоит из следующих модулей:

\begin{itemize}
  \item Scene --- модуль отвечающий за хранение данных о сцене и её объектах;
  \item Camera --- модуль для работы с камерой, вычисления направления взгляда,
    обработки движения камеры в пространстве;
  \item Shading --- модуль, отвечающий за расчет освещенности;
  \item Rendering --- модуль визуализации, выполняющий отрисовку сцены;
\end{itemize}

\subsection*{Вывод}

В данном разделе описаны алгоритмы, используемые для моделирования трехмерного фрактала
<<оболочка Мандельброта>>, его рендеринга и закраски.

\clearpage

