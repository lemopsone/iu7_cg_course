\section{Исследовательская часть}
В данном разделе представлены результаты исследования влияния параметров объектов сцены
и разрешения окна сцены на производительность
программы при визуализации фрактала <<оболочка Мандельброта>> разными методами рендеринга.
Производительность оценивалась по времени, необходимому для рендера одного кадра.

\subsection{Цель исследования}
Цель исследования --- определение зависимости производительности программы от
разрешения экрана и количества итераций при построении фрактала для всех реализованных
алгоритмов рендеринга.

\subsection{Методика исследования}
При проведении исследования для каждого из алгоритмов рендеринга создавалась сцена
с фиксированным количеством итераций для построения фрактала и фиксированным разрешением экрана,
после чего замерялось время, за которое программа создаст новый кадр.

Параметры исследования:
\begin{itemize}
  \item iterations --- количество итераций (изменяется от 1 до 15 с шагом 1);
  \item size --- размер стороны экрана в пикселях (изменяется от 200 до 1000 с шагом 100);
  \item renderer --- реализация алгоритма марширования лучей (однопоточная, многопоточная, с использованием OpenGL).
\end{itemize}

Для каждой исследуемой конфигурации сцены вида (iterations, size, renderer) проводилось
по 50 замеров, после чего результаты измерений усреднялись. Все измерения проводились на
одном устройстве.

\subsection{Технические характеристики}
Для проведения исследования использовалось устройство со следующими параметрами.

\begin{itemize}
  \item операционная система macOS Sequoia~\cite{macos};
  \item центральный процессор Apple M1 Pro, частота 3.2 ГГц, 10 ядер~\cite{macbook};
  \item графический процессор 16 ядер, пропускная способность 200 ГБ/с~\cite{macbook};
  \item оперативная память 16 ГБ LPDDR4X~\cite{macbook}.
\end{itemize}

\subsection{Результаты исследования}

\subsubsection{Влияние размеров сцены}

Результаты исследования зависимости времени отрисовки кадра от разрешения экрана представлены в таблице~\ref{tab:resolution} и на рисунках~\ref{img:resolution_comparison_linear}---\ref{img:resolution_comparison_log}.

\begin{table}[h]
  \begin{center}
    \captionsetup{justification=centering, margin=.5cm}
    \caption{Зависимость времени отрисовки кадра от разрешения экрана}\label{tab:resolution}
    \csvautotabular{inc/data/resolution_comparison.csv}
  \end{center}
\end{table}

\img{100mm}{resolution_comparison_linear}{Графики зависимости времени отрисовки кадра от разрешения экрана}

\img{100mm}{resolution_comparison_log}{Графики зависимости времени отрисовки кадра от разрешения экрана (логарифмическая шкала)}

\clearpage
\subsubsection{Влияние количества итераций фрактала}

Результаты исследования зависимости времени отрисовки кадра от количества итераций фрактала представлены в таблице~\ref{tab:iterations}
и на рисунках~\ref{img:iterations_comparison_linear_3}---\ref{img:iterations_comparison_log}.

\begin{table}[h]
  \begin{center}
    \captionsetup{justification=centering, margin=.5cm}
    \caption{Зависимость времени отрисовки кадра от количества итераций}\label{tab:iterations}
    \csvautotabular{inc/data/iterations_comparison.csv}
  \end{center}
\end{table}

\img{100mm}{iterations_comparison_linear_3}{Графики зависимости времени отрисовки кадра от количества итераций фрактала}

\img{100mm}{iterations_comparison_log}{Графики зависимости времени отрисовки кадра от количества итераций фрактала (логарифмическая шкала)}
\clearpage

\subsection{Вывод}
На основании проведённых исследований можно сделать следующие выводы.

\begin{enumerate}
  \item разрешение экрана оказывает наиболее существенное влияние на производительность программы;
  \item увеличение количества итераций изображаемого фрактала оказывает существенное влияние на производительность
    однопоточного алгоритма, однако для остальных алгоритмов его влияние не столь существенно;
  \item при любой конфигурации сцены версия алгоритма, выполняющая вычисления с помощью графического процессора,
    рендерит изображение сцены в десятки раз быстрее, чем однопоточная и многопоточная версии;
  \item многопоточная версия программы работает в среднем в 5-9 раз быстрее однопоточной;
\end{enumerate}

Таким образом, для оптимизации производительности следует использовать версию алгоритма марширования лучей,
использующую графический процессор для вычислений. При необходимости дальнейшей оптимизации
следует уменьшать разрешение экрана. Изменение количества итераций даст ощутимый прирост в производительности только
для однопоточного алгоритма, в остальных реализациях его влияние незначительно.
