\section{Аналитическая часть}

В данном разделе проводится анализ существующих алгоритмов для решения поставленной задачи. В результате анализа выбирается алгоритм для дальнейшей реализации.

\subsection{Анализ предметной области}
Понятия фрактал и фрактальная геометрия появились в конце 70-х годов XX века, и их основоположником
принято считать Бенуа Мандельброта, который в 1977 году выпустил книгу <<Фрактальная геометрия природы>>~\cite{mandelbrot}.
В ней ученый дает следующее определение фрактала.

Фрактал --- структура, состоящая из частей, которые в каком-то
смысле подобны целому. Слово фрактал образовано от латинского \textit{fractus} --- состоящий из фрагментов~\cite{mandelbrot}.

По общепринятой классификации фракталы делят на геометрические, алгебраические и стохастические~\cite{morozov}.

Геометрические фракталы (в двумерном случае) получают на основе некоторой исходной фигуры путём ее дробления и выполнения различных преобразований полученных фрагментов.

Одним из наиболее известных примеров геометрического фрактала является снежинка Коха. В качестве исходного объекта в ней выбран равносторонний треугольник со сторонами единичной длины.
На каждой итерации стороны фигуры делятся на 3 равные части, центральная часть отбрасывается, а на её месте строятся две боковые стороны равностороннего треугольника.

\img{30mm}{snowflake}{Снежинка Коха на разных итерациях}

Алгебраические фракталы получают с помощью итерационных процессов по выражениям вида
\begin{displaymath}
  Z_{n+1} = f(Z_{n}),
\end{displaymath}
где $z$ -- комплексное число, а $f$ -- некая функция.

Наиболее ярким представителем квадратичных алгебраических фракталов является
множество Мандельброта. Его получают на комплексной плоскости при итерировании по формуле
\begin{equation}
  Z_{n+1} = Z_{n}^2 + c,
  \label{eq:mandelbrot}
\end{equation}
где $c$ -- комплексная переменная.

\img{40mm}{mandelbrot}{Множество Мандельброта}

Стохастические фракталы получают при хаотическом изменении каких-либо параметров
в итерационном процессе. В отличие от алгебраических и геометрических фракталов, они не являются детерменированными.

Ярким примером стохастического фрактала является траектория броуновского движения на плоскости.

\img{40mm}{brown}{Траектория броуновского движения}

Новой ступенью развития фрактальной геометрии являются трёхмерные фракталы.

Оболочка Мандельброта --- трёхмерный фрактал, аналог множества Мандельброта, созданный
Д. Уайтом и П. Ниландером с использованием гиперкомплексной алгебры, основанной на сферических координатах.

Формула для n-ой степени трехмерного гиперкомплексного числа ${x, y, z}$ следующая:
\begin{equation}
  \langle x, y, z \rangle ^n = r^n \langle cos(n\theta)cos(n\phi), sin(n\theta)cos(n\phi), sin(n\phi) \rangle,
  \label{eq:hypercomplex}
\end{equation}
где

$\qquad\qquad\qquad\qquad\qquad r  = \sqrt{x^2 + y^2 + z^2};$

$\qquad\qquad\qquad\qquad\qquad \theta = arctan(y/x);$

$\qquad\qquad\qquad\qquad\qquad \phi = arctan(z/\sqrt{x^2 + y^2}) = arcsin(z/r);$

\qquad $r$ -- модуль, $\theta$ и $\phi$ -- аргументы гиперкомплексного числа.

В модели Уайта и Ниландера используется итерация $z \mapsto z^n + c$, где $z$ и $c$ ---
трехмерные гиперкомплексные числа~\cite{algebra_mandelbulb}, на которых операция возведения в натуральную степень
выполняется аналогично \ref{eq:hypercomplex}. Для $n > 3$, результатом
является трёхмерный фрактал.

\subsection{Формализация объектов синтезируемой сцены}

Сцена состоит из следующего набора объектов.
\begin{enumerate}
	\item точечный источник света --- задается положением в пространстве и интенсивностью; в зависимости от характеристик источника определяется освещенность объектов сцены; характеристики имеют значения по умолчанию, но имеется возможность их изменить;
	\item камера --- задается положением в пространстве, углом поворота вокруг каждой из трех координатных осей;
  \item изображаемый фрактал --- задаётся координатами центра объекта, степенью уравнения фрактала, количеством итераций для построения объекта.
\end{enumerate}

\subsection{Анализ методов рендера модели}
Рендеринг (\textit{англ. <<rendering>> --- <<визуализация>>}) --- термин, обозначающий
процесс получения изображения по модели с помощью компьютерной программы.

Фрактал <<оболочка Мандельброта>> является алгебраическим, т. е. описывается математическими уравнениями
и не имеет явной геометрии (например, треугольников или вокселей).

Рассмотрим основные методы рендера моделей.

\subsubsection{Растеризация}
Растровое изображение --- это изображение, представляющее собой сетку пикселей
-- цветных точек на мониторе, бумаге и других отображающих устройствах. % добавить цитату

Растеризация --- это процесс получения растрового изображения.

Технология основана на обходе лучем вершин треугольного полигона, который сохраняет
свою форму даже после попадания из трехмерного пространства в двухмерное.
Каждая точка объекта в трехмерном пространстве переводится в точку на экране, а затем
точки соединяются и получается изображение исходного объекта~\cite{rendering}.

Преимуществом метода является то, что современные компьютеры
оптимизированы для рендеринга растровых изображений, из-за чего процесс растеризации достаточно быстрый.

Недостатки:
\begin{itemize}
  \item изображение на выходе получается ступенчатым, требуется дополнительное сглаживание;
  \item метод работает с примитивами, следовательно требуется ресурсоемкая полигонализация сложных объектов;
  \item для моделей сцены могут использоваться миллионы полигонов, каждый из которых должен быть подвержен растеризации;
  \item каждый пиксель может обрабатываться множество раз (например, при наложении полигонов);
\end{itemize}

\subsubsection{Трассировка лучей}
Трассировка лучей (\textit{англ. <<ray tracing>>}) --- это технология отрисовки трехмерной графики,
симулирующая физическое поведение света~\cite{rt_rendering}. % цитата

Суть алгоритма заключается в моделировании пути света для создания изображения.
Лучи света, исходящие из камеры, проверяются на пересечение с полигонами и примитивами, принадлежащими объектам сцены.
Для каждого пересечения вычисляются параметры освещения, и испускаются вторичные лучи, которые используются для определения отражений и теней.

Преимущества:
\begin{itemize}
  \item вычислительная сложность метода слабо зависит от сложности сцены~\cite{rt_rendering};
  \item отсечение невидимых поверхностей, перспектива и корректное изменение поля зрения являются следствием алгоритма;
\end{itemize}

Недостатки:
\begin{itemize}
  \item алгоритм может работать только с примитивами, следовательно для сложных объектов необходимо проводить полигонализацию;
  \item метод имеет крайне высокую вычислительную сложность и является крайне ресурсоёмким~\cite{rt_rendering}.
\end{itemize}

\subsubsection{Марширование лучей}
Марширование лучей (\textit{англ. <<ray marching>>}) --- разновидность алгоритма трассировки лучей.

Как и трассировка лучей, этот метод испускает луч в сцену для каждого пикселя экрана.
Однако, в отличии от трассировщика лучей, данный метод не пытается вычислить пересечение луча и объекта аналитически.
В нём происзодит смещение текущего положения вдоль луча, пока не найдется точка, пересекающая объект.
Такая операция переноса является простой и нересурсозатратной, в отличие от аналитического вычисления пересечения,
однако данный способ является менее точным, чем обычная трассировка.

Для сцен, состоящих из непрозрачных объектов, можно ввести ещё одну оптимизацию в данный алгоритм.
Она заключается в использовании полей расстояний со знаком.

Поле расстояний со знаком (\textit{англ. <<signed distance field>>}) --- это функция,
получающая на вход точку пространства и возвращающая кратчайшее расстояние от этой точки
до поверхности каждого объекта в сцене~\cite{sdf}. Если точка находится внутри объекта, то функция возвращает отрицательное число.

Эта оптимизация позволяет заметно ограничить количество шагов при движении вдоль луча, что увеличивает эффективность метода.
\png{80mm}{raymarch}{Визуализация марширования лучей с использованием SDF. Красным отмечены все проверяемые точки}


Преимущества:
\begin{itemize}
  \item решает проблему производительности трассировки лучей за счёт оптимизаций;
  \item не требует разбиения поверхностей на примитивы, подходит для рендера сложных объектов;
  \item качество получаемого изображения сопоставимо с получаемым при трассировке лучей.
\end{itemize}

Недостаток метода заключается в том, что пересечение вычисляется менее точно,
чем при трассировке лучей, имеется некоторая погрешность.

\subsubsection{Сравнение алгоритмов}

Сравнение методов рендеринга будет проводиться по следующим критериям.
\begin{enumerate}
  \item качество изображения;
  \item эффективность по времени выполнения рендера;
  \item возможность рендерить сложные объекты без разбиения на примитивы.
\end{enumerate}

В таблице \ref{tbl:comp} приведено сравнение рассмотренных методов по указанным выше критериям.

\begin{table}[htbp]
	\centering
  \captionsetup{justification=centering}
	\caption{Сравнение алгоритмов}
	\label{tbl:comp}
  \begin{tabular}{|l|l|l|l|}
  \hline
  Метод        & Качество & Быстродействие & Сложные объекты \\ \hline
  Растеризация & 3        & 1              & --               \\ \hline
  Ray Tracing  & 1        & 3              & --               \\ \hline
  Ray Marching & 2        & 2              & +               \\ \hline
  \end{tabular}
\end{table}

Алгоритм марширования лучей подходит для рендера фракталов лучше прочих из рассмотренных,
так как он не требует разбиения модели на примитивы для работы, не требует хранения информации о вершинах и гранях объекта,
а также обладает высоким быстродействием.

\subsection{Анализ алгоритмов закраски}
\subsubsection{Простая закраска}
Алгоритм подразумевает однотонную закраску кажого многоугольника,
принадлежащего объекту, в зависимости от интенсивности в одной из его точек. При этом предполагается следующее:
\begin{itemize}
  \item источник света расположен в бесконечности;
  \item наблюдатель находится в бесконечности;
  \item многоугольник представляет собой реальную моделируемую поверхность,
    а не является аппроксимацией криволинейной поверхности.
\end{itemize}

Ключевым недостатком данного алгоритма является то, что все точки грани будут иметь одинаковую интенсивность.

\subsubsection{Метод закраски Гуро}
Метод Гуро позволяет устранить дискретность изменения интенсивности и создать иллюзию
гладкой криволинейной поверхности~\cite{rogers}. Он основан на интерполяции интенсивности. Сам алгоритм можно разбить на четыре этапа.
\begin{enumerate}
  \item вычисление нормалей ко всем полигонам объекта;
  \item определение нормали в вершинах путём усреднения нормалей по всем граням, принадлежащим
    рассматриваемым вершинам;
  \item вычисление значения интенсивности в вершинах на основе выбранной модели освещения;
  \item закраска каждого многоугольника путём линейной интерполяции в вершинах сначала вдоль ребер, а затем между ними.
\end{enumerate}

Достоинством метода является большая реалистичность получаемого изображения
по сравнению с простой закраской.

\subsubsection{Метод закраски Фонга}
При такой закраске, в отличие от метода закраски Гуро, интерполируется значение вектора нормали, а не интенсивности~\cite{phong}.
Алгоритм также имеет четыре основных этапа:
\begin{enumerate}
  \item вычисление векторов нормалей к каждой грани и к каждой вершине грани;
  \item интерполяция векторов нормалей вдоль ребер грани;
  \item линейная интерполяция векторов нормалей вдоль сканирующей строки;
  \item вычисление интенсивности в очередной точке сканирующей строки.
\end{enumerate}

Достоинства:
\begin{itemize}
  \item можно достичь лучшей аппроксимации кривизны поверхности;
  \item более реалистичные блики, чем в методе Гуро~\cite{rogers}.
\end{itemize}

Недостатки:
\begin{itemize}
  \item высокая ресурсоёмкость;
  \item высокая вычислительная сложность.
\end{itemize}

\subsubsection{Выбор алгоритма}
В качестве метода закраски был выбран метод Фонга, так как он позволяет получить лучшую
аппроксимацию сложной поверхности, чем простая закраска и метод закраски Гуро.

\subsection*{Вывод}
В данном разделе было проведено описание объектов сцены, рассмотрены алгоритмы визуализации сцены, отвечающие заданным требованиям.
В качестве алгоритма для рендера изображения был выбран алгоритм марширования лучей как
быстродейственный алгоритм, имеющий небольшую погрешность и позволяющий визуализировать сложные объекты сцены.
В качестве алгоритма закраски был выбран метод закраски Гуро, так как он является быстродейственным и позволяет достаточно качественно закрашивать объекты.

\clearpage
